\section{Graphene synthesis}
Recently, there are several methods used to synthesize graphene. After the first exfoliation of graphene \cite{firstExfoliation} several other techniques were developed to produce graphene sheets each with its own advantages and disadvantages \cite{mbayachi2021graphene}:



\subsubsection*{Mechanical exfoliation}
Mechanical exfoliation is also known as Scotch tape or peel-off method. It was the first method to be used by Novoselov and Geim for the production of graphene with the help of an adhesive tape to force the graphene layers apart, with this method flakes of on the order of $\mu$m$\sim$cm can be obtained \cite{firstExfoliation,edwards2013graphene,van2012production}. One the peeling process in completed, usually acetone is used to detach the flakes from the tape.
The main drawback of this process is that is slow and imprecise, hence the process is used to study the properties of graphene rather than using it commercially.

\subsubsection*{Electrochemical exfoliation}
This method involves the use of various forms of graphite such as graphite foils, plates, rods and graphite powders as electrodes in an aqueous or non-aqueous electrolyte and electric current to bring about the expansion of electrodes \cite{liu2019synthesis,yang2016new,luheng2009influence}.\\
The cathode is made of graphite, and a persistent voltage of $5$V are applied, after a few minutes the anode is covered by a thin layers of a black material, by cleaning the anode a graphene powder made of flakes of around $500\sim 700$nm can be obtained.


\subsubsection*{Pyrolysis}
The meaning of the word is \textit{separating with fire}.One of the common techniques of graphene synthesis is the thermal decomposition of silicon carbide (SiC). At high temperature, Si is desorbed leaving behind C atom which forms few graphene layers. The advantages of this method is the production of high purity graphene mono-layer up to cm size over the entire SiC coated surface, however it has proven challenging doing so at large scales \cite{shams2015synthesis,hibino2010graphene,juang2009synthesis,pan2009highly}.

\subsubsection*{Chemical vapor deposition}
Chemical vapor deposition (CVD) is a bottom-up synthesis technique used for production of high-quality graphene \cite{brownson2012electrochemistry}, this method involves combining a gas molecule with a surface substrate inside a reaction chamber. Different substrates are used in CVD for graphene film growth, they include Nickel (Ni)  \cite{yavari2011high} Copper (Cu) \cite{seekaew2017highly}, Iron (Fe) \cite{an2011graphene}, and Stainless steel \cite{ghaemi2016synthesis}. Methane (CH4) and acetylene (C2H2) are normally used as a carbon source. At high temperatures the hydrogen and carbon detach forming hydrogen gas and pure carbon, the carbon when it comes in contact with the substrate it attaches itself creating graphene. With this technique it is possible to create graphene flakes of the size of several cm, however scaling it up has proven challenging.