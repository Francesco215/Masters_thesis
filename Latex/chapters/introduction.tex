Suppose we start with a infinitely long strip of width $W$ made of a standard conducting material. If we apply a voltage $V$ at two opposite points of the strip a current will flow from one electrode to the other. The current will be the strongest along the segment that unites the electrodes, and will be exponentially weaker the further away it is. This off-axis current is called \textit{Nonlocal Current} and it also generates a voltage along the edges of the strip called \textit{Nonlocal Voltage}.\\
Experiments in high quality gapped graphene have highlighted the existence of a larger than expected nonlocal dc voltages.\\\\
The objective of this thesis is to first explain why this anomalous nonlocal current exists in gapped graphene, and develop a model that can be used to predict and analyze this kind of phenomenon.\\
The thesis is structured as follows:\\\\
The first chapter is devoted to explaining how some of the machinery we'll use work. This includes some of the main concepts of topology in solid state physics, like the Hall effect, Berry phase, Berry curvature. We then use the concept of Berry curvature to generalize the Hall effect and obtain the Kubo formula and the TKNN formula. We also give a quick overview of the electronic properties of graphene and derived like gapped graphene and bilayer graphene where we focus on what happens close to the Dirac points and we introduce the concept of \textit{Valley}.\\\\
The second chapter is devoted to see how all this tools come together in fact the nonlocal current arises from the Berry curvature hot spots near the Dirac points in gapped graphene which give rise to a particular kind of anomalous Hall effect called \textit{Valley Hall Effect}. This creates a transverse \textit{Valley Current} that is responsible for the high nonlocal voltage measured in the experiments\\\\
In the third and last chapter we create a model that explains in full detail how the voltage and currents behave inside of a graphene strip and develop an approximation that can be used to evaluate the properties of these kind of system and apply it to data from real-world experiments.